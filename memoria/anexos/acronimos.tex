%%%%%%%%%%%%%%%%%%%%%%%%%%%%%%%%%%%%%%%%%%%%%%%%%%%%%%%%%%%%%%%%%%%%%%%%
% Plantilla TFG/TFM
% Escuela Politécnica Superior de la Universidad de Alicante
% Realizado por: Jose Manuel Requena Plens
% Contacto: info@jmrplens.com / Telegram:@jmrplens
%%%%%%%%%%%%%%%%%%%%%%%%%%%%%%%%%%%%%%%%%%%%%%%%%%%%%%%%%%%%%%%%%%%%%%%%

% Lista de acrónimos (se ordenan por orden alfabético automáticamente)

% La forma de definir un acrónimo es la siguiente:
% \newacronym{id}{siglas}{descripción}
% Donde:
% 	'id' es como vas a llamarlo desde el documento.
%	'siglas' son las siglas del acrónimo.
%	'descripción' es el texto que representan las siglas.
%
% Para usarlo en el documento tienes 4 formas:
% \gls{id} - Añade el acrónimo en su forma larga y con las siglas si es la primera vez que se utiliza, el resto de veces solo añade las siglas. (No utilices este en títulos de capítulos o secciones).
% \glsentryshort{id} - Añade solo las siglas de la id
% \glsentrylong{id} - Añade solo la descripción de la id
% \glsentryfull{id} - Añade tanto  la descripción como las siglas

\newacronym{ieee}{IEEE}{Institute of Electrical and Electronics Engineers}
\newacronym{tfg}{TFG}{Trabajo Final de Grado}
\newacronym{tfm}{TFM}{Trabajo Final de Máster}
\newacronym{apa}{APA}{American Psychological Association}
\newacronym{asa}{ASA}{Acoustical Society of America}
\newacronym{adaa}{ADAA}{Asociación de Acústicos Argentinos}
\newacronym{aes}{AES}{Audio Engineering Society}
\newacronym{aas}{AAS}{Australian Acoustical Society}
\newacronym{csic}{CSIC}{Consejo Superior de Investigaciones Científicas}
\newacronym{eaa}{EAA}{European Acoustics Association}
\newacronym{ioa}{IOA}{Institute Of Acoustics}
\newacronym{ica}{ICA}{International Congress on Acoustics}
\newacronym{iiav}{IIAV}{International Institute of Acoustics and Vibration}
\newacronym{ince}{I-INCE}{International Institute of Noise Control Engineering}
\newacronym{isva}{ISVA}{International Seminar on Virtual Acoustics}
\newacronym{isra}{ISRA}{International Symposium on Room Acoustics}
\newacronym{sea}{SEA}{Sociedad Española de Acústica}
\newacronym{ia}{IA}{Inteligencia Artificial}
\newacronym{rag}{RAG}{Retrieval-Augmented Generation (Generación Aumentada por Recuperación)}
\newacronym{pln}{PLN}{Procesamiento del Lenguaje Natural}
\newacronym{llm}{LLM}{Large Language Model}
\newacronym{gpt}{GPT}{Generative Pre-trained Transformer}
\newacronym{clip}{CLIP}{Contrastive Language-Image Pre-training}
\newacronym{eps}{EPS}{Escuela Politécnica Superior} 
\newacronym{une}{UNE}{Una Norma Española} 