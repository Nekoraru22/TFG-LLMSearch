%%%%%%%%%%%%%%%%%%%%%%%%%%%%%%%%%%%%%%%%%%%%%%%%%%%%%%%%%%%%%%%%%%%%%%%%
% Plantilla TFG/TFM
% Escuela Politécnica Superior de la Universidad de Alicante
% Realizado por: Jose Manuel Requena Plens
% Contacto: info@jmrplens.com / Telegram:@jmrplens
%%%%%%%%%%%%%%%%%%%%%%%%%%%%%%%%%%%%%%%%%%%%%%%%%%%%%%%%%%%%%%%%%%%%%%%%

% \chapter{Anexo I}
\chapter{Script de Prueba para ChromaDB}
\label{anx:chroma_script}

A continuación, se presenta el script de Python utilizado para las pruebas de concepto con ChromaDB, detalladas en la Sección \ref{sec:evaluacion}.

\begin{lstlisting}[language=Python, caption={Script de Python para la prueba de concepto con ChromaDB.}, label={lst:chroma_prove_code}]
import numpy as np
import matplotlib.pyplot as plt
from chromadb import Client as ChromaClient # Asumo que así importas tu cliente

# Definición de la clase ChromaClient o funciones si no es una clase estándar
# Si ChromaClient es una clase que tú has definido, asegúrate que esté disponible
# o que las funciones que usa (create_embeddings, etc.) estén definidas.
# Por simplicidad, voy a asumir que tu `ChromaClient` ya tiene esos métodos.
# Si `ChromaClient` es de la librería `chromadb`, entonces el import es suficiente.

def prove() -> None:
    # Usar una ruta relativa o absoluta que funcione en tu entorno
    chroma = ChromaClient(settings={"persist_directory": "./data/chroma_prove_db",
                                    "chroma_db_impl": "duckdb+parquet"}) # Ejemplo de settings si es necesario

    # Create some example documents
    documentos = [
        "Python is a high-level, interpreted programming language",
        "Embeddings are vector representations of text",
        "Chroma is a vector database for storing embeddings",
        "Language models can generate semantic embeddings",
        "3D visualization helps to understand the distance between embeddings",
        "Vector databases are useful for semantic searches",
        "Embeddings capture the semantics of words and phrases",
        "Python has many libraries for natural language processing"
    ]
    
    # Crear embeddings para los documentos
    # Esta parte depende de cómo tu ChromaClient o la librería chromadb genera embeddings.
    # Si usas el modelo por defecto de la librería:
    # (No necesitas llamar a create_embeddings si la librería lo hace internamente al añadir)
    # Para este ejemplo, asumiré que tienes un método o que la librería lo maneja.
    # Si `chroma.create_embeddings` no existe, deberás usar el método correcto
    # de la librería `chromadb` para obtener los embeddings, ej. un EmbeddingFunction.

    # Crear una colección en ChromaDB (esto también podría crearla si no existe al añadir documentos)
    collection_name = "example_embeddings"
    try:
        collection = chroma.get_collection(name=collection_name)
    except: # Ajusta la excepción específica si es necesario
        collection = chroma.create_collection(name=collection_name)
    
    # Añadir documentos a la colección.
    # ChromaDB típicamente requiere IDs para los documentos.
    ids = [f"doc{i}" for i in range(len(documentos))]
    
    # Si ChromaDB genera los embeddings automáticamente al añadir, no necesitas pasarlos explícitamente.
    # Si SÍ necesitas pasar embeddings pre-calculados, necesitarías una función para ello.
    # Este ejemplo asume que la librería puede manejar los embeddings directamente o con una
    # función de embedding configurada al crear la colección/cliente.
    collection.add(
        documents=documentos,
        ids=ids
    )
    
    # Realizar una búsqueda
    query = "What are embeddings?"
    resultados = collection.query(
        query_texts=[query],
        n_results=3,
        include=['documents', 'distances'] # Asegúrate de incluir lo que necesitas
    )
    
    print("\nSearch results for:", query)
    if resultados['documents'] is not None and len(resultados['documents'][0]) > 0:
        for i, doc in enumerate(resultados['documents'][0]):
            distance = resultados['distances'][0][i] if resultados['distances'] else 'N/A'
            print(f"{i+1}. {doc} (Distance: {distance:.4f})")
    else:
        print("No similar documents were found.")
    
    # Para la visualización 3D y matriz de distancias, necesitarías obtener todos los embeddings
    # y el embedding de la consulta. La librería `chromadb` puede que no ofrezca estas
    # visualizaciones directamente como `chroma.visualize_embeddings_3d`.
    # Estas visualizaciones suelen hacerse con librerías como matplotlib, plotly, scikit-learn (para PCA/t-SNE).

    # Obtener todos los embeddings de la colección (si es necesario para visualización manual)
    all_docs_data = collection.get(include=['embeddings', 'documents'])
    stored_embeddings = np.array(all_docs_data['embeddings'])
    stored_documents_text = all_docs_data['documents']
    
    labels = [f"Doc {i}: {doc[:20]}..." for i, doc in enumerate(stored_documents_text)]
    
    # Si quieres visualizar la consulta también, necesitas su embedding
    # Esto depende de cómo se obtienen los embeddings (ej. usando la misma embedding function)
    # Para el ejemplo, no se incluye la visualización 3D/matriz compleja aquí
    # ya que requeriría más código para PCA/t-SNE y plotteo con matplotlib/plotly
    # que no está en tu `prove()` original de forma explícita con la librería `chromadb`.
    
    # Si las funciones `visualize_embeddings_3d` y `visualize_matriz_distances`
    # eran parte de TU clase `ChromaClient` personalizada, tendrías que incluirlas.
    # Las librerías estándar de ChromaDB no suelen tener estas funciones de visualización directa.

    # plt.show() # Solo si generas figuras con matplotlib

    # print("\nDistance Matrix (conceptual):")
    # (Calcular y mostrar la matriz de distancias requeriría sklearn.metrics.pairwise_distances por ejemplo)

if __name__ == '__main__':
    prove()
\end{lstlisting}