%%%%%%%%%%%%%%%%%%%%%%%%%%%%%%%%%%%%%%%%%%%%%%%%%%%%%%%%%%%%%%%%%%%%%%%%
% Plantilla TFG/TFM
% Escuela Politécnica Superior de la Universidad de Alicante
% Realizado por: Jose Manuel Requena Plens
% Contacto: info@jmrplens.com / Telegram:@jmrplens
%%%%%%%%%%%%%%%%%%%%%%%%%%%%%%%%%%%%%%%%%%%%%%%%%%%%%%%%%%%%%%%%%%%%%%%%

\chapter{Estado del Arte}
\label{marcoteorico}

La construcción de un sistema inteligente para la búsqueda y gestión de archivos personales requiere la integración de diversas tecnologías y herramientas consolidadas en el ámbito del desarrollo de software y la inteligencia artificial. Este capítulo tiene como objetivo revisar el estado del arte de los componentes tecnológicos clave que se han considerado o que forman la base para la implementación del presente proyecto. Se analizarán diferentes opciones en áreas fundamentales como la orquestación de tareas, la detección de cambios en el sistema de archivos, las soluciones de bases de datos para el almacenamiento de metadatos y embeddings, la contenerización para el despliegue y, finalmente, los frameworks para el desarrollo de la interfaz de usuario. Esta revisión permitirá contextualizar las decisiones de diseño tomadas y justificar la selección de las herramientas específicas utilizadas en el desarrollo de la solución.

\section{Orquestador de tareas}
Herramientas para la automatización y gestión de flujos de trabajo.

\subsection{Prefect}
Plataforma de orquestación de flujos de trabajo en Python que permite diseñar, ejecutar y monitorizar pipelines de datos y machine learning de forma fiable y escalable.

\subsubsection{Ventajas}
\begin{itemize}
    \item \textbf{Facilidad de uso:} Prefect ofrece una sintaxis intuitiva y una configuración sencilla, lo que facilita la definición y gestión de flujos de trabajo complejos.
    
    \item \textbf{Flexibilidad:} Permite la orquestación de tareas en entornos locales, en la nube o híbridos, adaptándose a diversas necesidades.
    
    \item \textbf{Monitoreo y gestión:} Incluye herramientas integradas para el monitoreo, registro y manejo de errores en tiempo real.
\end{itemize}

\subsubsection{Desventajas}
\begin{itemize}
    \item \textbf{Madurez:} Aunque ha ganado popularidad, Prefect es relativamente nuevo en comparación con otras herramientas más consolidadas.
    
    \item \textbf{Comunidad:} Su comunidad es más pequeña, lo que puede limitar la disponibilidad de recursos y soporte.
\end{itemize}

\subsection{Kafka}
Sistema de mensajería distribuido de alto rendimiento.

\subsubsection{Ventajas}
\begin{itemize}
    \item \textbf{Alto rendimiento:} Kafka es conocido por su capacidad para manejar grandes volúmenes de datos con baja latencia.
    
    \item \textbf{Escalabilidad:} Diseñado para escalar horizontalmente, puede manejar cargas de trabajo crecientes de manera eficiente.
    
    \item \textbf{Ecosistema robusto:} Cuenta con una amplia gama de herramientas y conectores que facilitan su integración con otros sistemas.
\end{itemize}

\subsubsection{Desventajas}
\begin{itemize}
    \item \textbf{Complejidad:} La configuración y gestión de Kafka pueden ser complejas, especialmente para usuarios sin experiencia previa.
    
    \item \textbf{Requisitos de recursos:} Para un rendimiento óptimo, Kafka suele requerir una infraestructura robusta, lo que puede ser excesivo para proyectos más pequeños.
\end{itemize}

\subsection{Airflow}
Plataforma para crear, programar y monitorear flujos de trabajo.

\subsubsection{Ventajas}
\begin{itemize}
    \item \textbf{Popularidad y comunidad:} Amplia adopción y una comunidad activa que proporciona numerosos recursos y soporte.
    
    \item \textbf{Flexibilidad:} Permite la programación y monitoreo de flujos de trabajo complejos.
\end{itemize}

\subsubsection{Desventajas}
\begin{itemize}
    \item \textbf{Curva de aprendizaje:} Puede ser complejo de configurar y requiere conocimientos avanzados para su implementación efectiva.
\end{itemize}

\clearpage
\section{Script de detección de cambios en un path y Servidor}
Herramientas para monitorear cambios en el sistema de archivos.

\subsection{Python}
Lenguaje de programación versátil con varias bibliotecas para detección de cambios.

\begin{itemize}
    \item \textbf{Watchdogs:} Biblioteca multiplataforma diseñada específicamente para detectar eventos en el sistema de archivos.
    
    \item \textbf{pyinotify:} Biblioteca que proporciona monitoreo de eventos del sistema de archivos en sistemas Linux aprovechando el subsistema inotify. Es eficiente para entornos Linux pero no es multiplataforma.
    
    \item \textbf{inotify-simple:} Un wrapper sencillo alrededor de la API inotify de Linux, que ofrece simplicidad y facilidad de uso para tareas básicas de monitoreo de archivos en sistemas Linux.
    
    \item \textbf{inotifyx:} Similar a pyinotify, esta biblioteca proporciona acceso al sistema inotify de Linux, permitiendo monitorear eventos del sistema de archivos. Está diseñada para tener una API estable pero también está limitada a plataformas Linux.
    
    \item \textbf{Polling Methods:} Para plataformas donde inotify no está disponible, o para requisitos más simples, implementar un mecanismo de sondeo puede ser una alternativa viable. Esto implica verificar periódicamente el sistema de archivos para detectar cambios.
\end{itemize}

\subsection{Node}
Entorno de ejecución para JavaScript con opciones para monitoreo de archivos.

\begin{itemize}
    \item \textbf{chokidar:} Biblioteca eficiente para vigilar cambios en el sistema de archivos.
\end{itemize}

\subsection{Java}
Lenguaje de programación con APIs nativas para monitoreo.

\begin{itemize}
    \item \textbf{WatchService:} API integrada en Java para monitorear cambios en directorios.
\end{itemize}

\subsection{C++/C/C\#}
Familia de lenguajes con herramientas para vigilancia de archivos.

\begin{itemize}
    \item \textbf{FileSystemWatcher:} Clase para monitorear cambios en el sistema de archivos.
\end{itemize}

\subsection{Go}
Lenguaje de programación con bibliotecas específicas.

\begin{itemize}
    \item \textbf{fsnotify:} Biblioteca Go para monitoreo del sistema de archivos.
\end{itemize}

\subsection{Rust}
Lenguaje de programación moderno con enfoque en seguridad.

\begin{itemize}
    \item \textbf{notify:} Biblioteca Rust para vigilar cambios en archivos.
\end{itemize}

\clearpage
\section{Base de datos}
Opciones para almacenamiento de datos.

\subsection{Relacional}
Bases de datos con estructura definida y relaciones entre tablas.

\subsubsection{SQLite}
Base de datos relacional ligera contenida en un único archivo.

\paragraph{Ventajas}
\begin{itemize}
    \item \textbf{Ligereza y simplicidad:} SQLite es una biblioteca de base de datos integrada que no requiere una configuración de servidor independiente, lo que facilita su implementación y uso.
    
    \item \textbf{Portabilidad:} Al almacenar toda la base de datos en un único archivo, es fácil de transferir y gestionar, especialmente útil para aplicaciones móviles o integradas.
    
    \item \textbf{Rendimiento en entornos de bajo recurso:} Funciona eficientemente incluso en sistemas con recursos limitados.
\end{itemize}

\paragraph{Desventajas}
\begin{itemize}
    \item \textbf{Concurrencia limitada:} SQLite permite múltiples lecturas simultáneas, pero las escrituras se gestionan de una en una, lo que puede ser un cuello de botella en aplicaciones con alta concurrencia de escritura.
    
    \item \textbf{Escalabilidad:} No está diseñada para manejar grandes volúmenes de datos o aplicaciones que requieren escalabilidad horizontal.
\end{itemize}

\subsubsection{MariaDB}
Sistema de gestión de bases de datos fork de MySQL.

\paragraph{Ventajas}
\begin{itemize}
    \item \textbf{Rendimiento y escalabilidad:} MariaDB ofrece un alto rendimiento y puede manejar una gran cantidad de transacciones, siendo adecuada para aplicaciones empresariales.
    
    \item \textbf{Compatibilidad con MySQL:} Como un fork de MySQL, mantiene una alta compatibilidad, facilitando la migración desde MySQL.
    
    \item \textbf{Soporte para almacenamiento en columnas:} Incluye el motor ColumnStore, optimizado para cargas de trabajo analíticas.
\end{itemize}

\paragraph{Desventajas}
\begin{itemize}
    \item \textbf{Complejidad en la configuración:} Requiere una configuración y gestión más complejas en comparación con SQLite.
    
    \item \textbf{Requisitos de recursos:} Necesita más recursos del sistema, lo que puede ser excesivo para aplicaciones pequeñas o integradas.
\end{itemize}

\subsection{No relacional}
Bases de datos con esquemas flexibles no basados en tablas relacionales.

\subsubsection{MongoDB}
Base de datos NoSQL orientada a documentos.

\paragraph{Ventajas}
\begin{itemize}
    \item \textbf{Flexibilidad del esquema:} Al ser una base de datos NoSQL orientada a documentos, permite almacenar datos en un formato flexible similar a JSON, adaptándose fácilmente a cambios en la estructura de los datos.
    
    \item \textbf{Escalabilidad horizontal:} Diseñada para escalar horizontalmente mediante sharding, lo que facilita el manejo de grandes volúmenes de datos y altas tasas de tráfico.
    
    \item \textbf{Alto rendimiento en operaciones de lectura/escritura:} Optimizada para manejar operaciones simultáneas de lectura y escritura de manera eficiente.
\end{itemize}

\paragraph{Desventajas}
\begin{itemize}
    \item \textbf{Consumo de recursos:} Requiere una cantidad significativa de recursos, especialmente en implementaciones a gran escala.
    
    \item \textbf{Falta de soporte para transacciones complejas:} Aunque MongoDB ha mejorado en este aspecto, las transacciones en múltiples documentos pueden no ser tan robustas como en bases de datos relacionales.
\end{itemize}

\clearpage
\section{Docker}
Plataforma de contenedores para desarrollo, envío y ejecución de aplicaciones.

\subsection{Ventajas}
\begin{itemize}
    \item \textbf{Portabilidad:} Los contenedores Docker aseguran que el software se ejecute de manera consistente en cualquier entorno.
    
    \item \textbf{Aislamiento:} Cada componente del sistema puede ejecutarse en su propio contenedor, evitando conflictos de dependencias.
    
    \item \textbf{Facilidad de despliegue:} Simplifica la distribución y actualización de aplicaciones.
\end{itemize}

\subsection{Desventajas}
\begin{itemize}
    \item \textbf{Consumo de recursos:} Aunque ligero, el uso de contenedores añade una capa adicional que consume recursos del sistema.
    
    \item \textbf{Complejidad adicional:} Requiere conocimientos sobre Docker y la gestión de contenedores.
\end{itemize}

\clearpage
\section{Interfaz}
Frameworks para el desarrollo de interfaces web.

\subsection{Angular}
Framework completo para desarrollo de aplicaciones web.

\subsubsection{Ventajas}
\begin{itemize}
    \item \textbf{Framework completo:} Angular ofrece una solución integral con herramientas integradas para el desarrollo de aplicaciones web robustas.
    
    \item \textbf{Arquitectura estructurada:} Facilita la escalabilidad y el mantenimiento de aplicaciones complejas.
\end{itemize}

\subsubsection{Desventajas}
\begin{itemize}
    \item \textbf{Curva de aprendizaje pronunciada:} Requiere tiempo para dominar conceptos como TypeScript y la inyección de dependencias.
    
    \item \textbf{Complejidad innecesaria para proyectos simples:} Puede ser excesivo para aplicaciones con funcionalidades limitadas.
\end{itemize}

\subsection{React}
Biblioteca de JavaScript para construcción de interfaces de usuario.

\subsubsection{Ventajas}
\begin{itemize}
    \item \textbf{Biblioteca flexible:} React se centra en la construcción de interfaces de usuario, permitiendo la integración con diversas bibliotecas según las necesidades del proyecto.
    
    \item \textbf{Amplia comunidad y recursos:} Su popularidad asegura una gran cantidad de recursos y soporte.
\end{itemize}

\subsubsection{Desventajas}
\begin{itemize}
    \item \textbf{Necesidad de configuraciones adicionales:} Para funcionalidades más allá de la UI, es necesario integrar bibliotecas adicionales, lo que puede aumentar la complejidad.
\end{itemize}

\subsection{Vue}
Framework progresivo para construir interfaces de usuario.

\subsubsection{Ventajas}
\begin{itemize}
    \item \textbf{Simplicidad y facilidad de uso:} Vue es conocido por su curva de aprendizaje suave, lo que permite una adopción rápida.
    
    \item \textbf{Flexibilidad:} Adecuado tanto para proyectos pequeños como para aplicaciones más complejas.
\end{itemize}

\subsubsection{Desventajas}
\begin{itemize}
    \item \textbf{Menor adopción en grandes empresas:} Aunque está ganando popularidad, Vue aún no es tan común en entornos corporativos grandes.
\end{itemize}

\subsection{Astro}
Framework moderno para sitios web con enfoque en rendimiento.

\subsubsection{Ventajas}
\begin{itemize}
    \item \textbf{Optimización para contenido estático:} Astro está diseñado para generar sitios web estáticos rápidos, lo que puede ser beneficioso para aplicaciones con contenido predominantemente estático.
    
    \item \textbf{Integración con otros frameworks:} Permite utilizar componentes de React, Vue y otros dentro de sus proyectos.
\end{itemize}

\subsubsection{Desventajas}
\begin{itemize}
    \item \textbf{Menor madurez:} Al ser relativamente nuevo, Astro puede carecer de la amplitud de recursos y comunidad que tienen otros frameworks más establecidos.
\end{itemize}