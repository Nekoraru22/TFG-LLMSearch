%%%%%%%%%%%%%%%%%%%%%%%%%%%%%%%%%%%%%%%%%%%%%%%%%%%%%%%%%%%%%%%%%%%%%%%%
% Plantilla TFG/TFM
% Escuela Politécnica Superior de la Universidad de Alicante
% Realizado por: Jose Manuel Requena Plens
% Contacto: info@jmrplens.com / Telegram:@jmrplens
%%%%%%%%%%%%%%%%%%%%%%%%%%%%%%%%%%%%%%%%%%%%%%%%%%%%%%%%%%%%%%%%%%%%%%%%

\chapter{Objetivos}
\label{objetivos}

\section{Objetivo general}
El objetivo principal de este \gls{tfg} es diseñar y desarrollar \textbf{un prototipo} de un buscador multimedia inteligente que permita a los usuarios realizar búsquedas avanzadas utilizando lenguaje natural. De esta manera, el usuario podrá localizar documentos de texto, imágenes, vídeos o archivos de audio, entre otros, buscando por el contenido intrínseco de los archivos para no acabar limitados por las búsquedas basadas únicamente en metadatos o en el contenido total.

En caso de querer llevar este prototipo a producción haría falta realizar un estudio mucho más intenso centro sobre todo en modelos optimizados para dispositivos móviles, así como en la optimización de la base de datos y el sistema de búsqueda. Este \gls{tfg} se centra en la creación de un prototipo funcional que demuestre la viabilidad del enfoque propuesto y sirva como base para futuras investigaciones y desarrollos en el campo de la búsqueda multimedia inteligente.

La idea es crear una herramienta que facilite a los usuarios encontrar contenido multimedia de manera eficiente y precisa mediante descripciones detalladas en lenguaje natural. Por ejemplo, se podría buscar una fotografía específica entre miles con una consulta como: “busca una foto en la que salía un gato naranja durmiendo sobre un sofá de cuero y que la hice en Japón hace unos 5 o 6 años”; o encontrar un archivo PDF relevante mediante una búsqueda del tipo: “encuentra los datos para la declaración de la renta de 2020”.
De esta forma, se pretende obtener un sistema de búsqueda que no solo identifique el archivo específico que se busca, sino que también tenga la capacidad de extraer datos relevantes del contenido del archivo para responder a preguntas específicas formuladas en la consulta, aprovechando las capacidades de los modelos de lenguaje aumentados por recuperación (RAG).

Más específicamente, el sistema LLMSearch resultante deberá ser capaz de procesar un conjunto de archivos locales proporcionados por el usuario como archivos PDF y TXT, imágenes en formatos JPEG y PNG, y archivos de audio/vídeo en formatos MP3/MP4 para generar embeddings multimodales. Estos embeddings se almacenarán en una base de datos vectorial optimizada para búsquedas de similitud. La interacción con el usuario se realizará a través de una interfaz gráfica simple e intuitiva que permitirá hacer consultas en lenguaje natural. El sistema, utilizando una arquitectura \gls{rag}, recuperará los documentos más relevantes y devolverá el path los documentos más relevantes junto a una pequeña descripción si se le especifica.

También, se busca que el sistema sea lo suficientemente flexible y escalable para permitir la integración de nuevos tipos de archivos y modelos de lenguaje en el futuro, así como la posibilidad de realizar búsquedas más complejas o específicas, de manera que se pueda ejecutar en un servidor con muchos recursos pero también en un teléfono móvil o un ordenador portátil de gama baja-media.

\section{Objetivos secundarios}
Además, se plantean los siguientes objetivos secundarios que complementan y dan soporte al objetivo principal:
\subsection{Estudiar modelos multimodales}
Estudiar diferentes modelos multimodales con el fin de seleccionar aquellos que ofrezcan los mejores resultados en términos de precisión y eficiencia (tiempo de respuesta razonable).
\subsection{Seleccionar una solución de base de datos}
Investigar y seleccionar una solución de base de datos adecuada para el almacenamiento y consulta eficiente de metadatos enriquecidos y embeddings vectoriales generados por los modelos de \gls{ia}.
\subsection{Diseñar una arquitectura modular}
Diseñar una arquitectura de sistema que sea modular, escalable y eficiente, permitiendo la integración de los diferentes componentes y facilitando futuras expansiones o mejoras.
\subsection{Desarrollar una interfaz gráfica}
Desarrollar una interfaz gráfica de usuario (GUI) intuitiva y amigable que permita a los usuarios interactuar fácilmente con el sistema, realizar búsquedas, visualizar los resultados obtenidos y gestionar sus archivos.
