%%%%%%%%%%%%%%%%%%%%%%%%%%%%%%%%%%%%%%%%%%%%%%%%%%%%%%%%%%%%%%%%%%%%%%%%
% Plantilla TFG/TFM
% Escuela Politécnica Superior de la Universidad de Alicante
% Realizado por: Jose Manuel Requena Plens
% Contacto: info@jmrplens.com / Telegram:@jmrplens
%%%%%%%%%%%%%%%%%%%%%%%%%%%%%%%%%%%%%%%%%%%%%%%%%%%%%%%%%%%%%%%%%%%%%%%%

\chapter{Objetivos}
\label{objetivos}

El objetivo de este proyecto es diseñar y desarrollar un buscador multimedia que permita a los usuarios realizar búsquedas avanzadas utilizando lenguaje natural de manera que, el usuario pueda buscar documentos de texto, imágenes, vídeos o archivos de audio usando el lenguaje natural para explorar el contenido de los ficheros, no solo los metadatos de los archivos.
La idea es crear una herramienta que permita a los usuarios encontrar contenido multimedia de manera eficiente y precisa utilizando descripciones detalladas en lenguaje natural. Por ejemplo, se podría buscar una fotografía entre miles con una descripción como “busca una foto en la que salía un elefante levantando la trompa y que la hice en Tailandia hace unos 5 o 6 años” o encontrar un archivo PDF con una búsqueda del tipo “encuentra los datos para la declaración de la renta de 2016”.
De esta forma, se obtendría un buscador que además de encontrar el archivo específico que se busca podría extraer los datos relevantes del archivo para responder a preguntas específicas hechas en la consulta.