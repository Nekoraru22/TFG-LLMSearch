%%%%%%%%%%%%%%%%%%%%%%%%%%%%%%%%%%%%%%%%%%%%%%%%%%%%%%%%%%%%%%%%%%%%%%%%
% Plantilla TFG/TFM
% Escuela Politécnica Superior de la Universidad de Alicante
% Realizado por: Jose Manuel Requena Plens
% Contacto: info@jmrplens.com / Telegram:@jmrplens
%%%%%%%%%%%%%%%%%%%%%%%%%%%%%%%%%%%%%%%%%%%%%%%%%%%%%%%%%%%%%%%%%%%%%%%%

\chapter*{Preámbulo}
\thispagestyle{empty}
\begin{quote}
Este proyecto surge por dos motivos. Por un lado, hay un interés en entender mejor cómo se usan y configuran las inteligencias artificiales multimodales. Por otro lado, se observa un problema común en la forma en que manejamos la información digital: la dificultad para encontrar archivos concretos (como imágenes o documentos) cuando se tiene una gran cantidad de datos. Esta misma situación se da en contextos más actuales, como al buscar elementos específicos, por ejemplo, ”stickers”, en aplicaciones de mensajería tipo WhatsApp o Telegram.
\end{quote}


\cleardoublepage %salta a nueva página impar
\chapter*{Agradecimientos}
\thispagestyle{empty}
Este rinconcito es para vosotros, para toda esa gente increíble que ha hecho posible que hoy esté aquí, escribiendo estas líneas.

Primero, a ese montón de compañeros y amigos que me he cruzado en la carrera. He aprendido un millón de cosas con vosotros, no solo de manera académica, sino lecciones de vida que se quedan para siempre. Sin vuestras risas, vuestros ánimos en los momentos de bajón y esa forma de tirar para adelante juntos, no sé si habría encontrado las fuerzas para seguir tantas veces. Sois, en gran parte, la razón de que esté celebrando este logro.

A mi familia, mi pilar fundamental. Gracias por creer en mí incluso cuando yo dudaba, por ponerme las cosas fáciles y por todo el apoyo para que pudiera dedicarme a esto. Sois increíbles. Y un gracias enorme y especial para mi hermano, Abel Gandía Ruiz. Tú fuiste quien me abrió los ojos a este mundo tan interesante de la programación cuando yo no tenía ni idea, quien me animó y me echó una mano para empezar.

También quiero acordarme de mis profes. Algunos habéis sido una inspiración, de esos que te contagian la ilusión y te hacen descubrir la magia en sitios donde nunca te lo hubieras imaginado. Gracias por inspirarme y ayudarme a ser un mejor ingeniero.

Y, cómo no, a mi tutor, Iván Gadea Sáez. Gracias por guiarme con este proyecto, por tu paciencia infinita y por ayudarme a calmar todos los nervios y dudas que me han ido surgiendo.

De verdad, a todos y cada uno, ¡muchísimas gracias!

\cleardoublepage %salta a nueva página impar
% Aquí va la dedicatoria si la hubiese. Si no, comentar la(s) linea(s) siguientes
\chapter*{}
\setlength{\leftmargin}{0.5\textwidth}
\setlength{\parsep}{0cm}
\addtolength{\topsep}{0.5cm}
\begin{flushright}
\small\em{
A quienes me inspiraron a soñar y a programar, recordándome que,\\
si puedo imaginarlo, puedo crearlo. \footnote{Alejandro Taboada, creador del canal "Programación ATS"}
}
\end{flushright}
