%%%%%%%%%%%%%%%%%%%%%%%%%%%%%%%%%%%%%%%%%%%%%%%%%%%%%%%%%%%%%%%%%%%%%%%%
% Plantilla TFG/TFM
% Escuela Politécnica Superior de la Universidad de Alicante
% Realizado por: Jose Manuel Requena Plens
% Contacto: info@jmrplens.com / Telegram:@jmrplens
%%%%%%%%%%%%%%%%%%%%%%%%%%%%%%%%%%%%%%%%%%%%%%%%%%%%%%%%%%%%%%%%%%%%%%%%

\chapter*{Preámbulo}
\thispagestyle{empty}
\begin{quote}
Este proyecto surge por dos motivos. Por un lado, hay un interés en entender mejor cómo se usan y configuran las inteligencias artificiales multimodales. Por otro lado, se observa un problema común en la forma en que manejamos la información digital: la dificultad para encontrar archivos concretos (como imágenes o documentos) cuando se tiene una gran cantidad de datos. Esta misma situación se da en contextos más actuales, como al buscar elementos específicos, por ejemplo, ”stickers”, en aplicaciones de mensajería tipo WhatsApp o Telegram.
\end{quote}

\cleardoublepage %salta a nueva página impar
\chapter*{Resumen}
\thispagestyle{empty}
Este Trabajo de Fin de Grado con nombre “LLMSearch: Buscador multimedia basado en lenguaje natural”, consiste en desarrollar un sistema que permita la búsqueda de archivos multimedia mediante consultas formuladas en lenguaje natural como si se le preguntase a una persona. La motivación del proyecto surge de la dificultad para localizar archivos específicos dentro de grandes volúmenes de datos, en especial cuando el usuario solo recuerda detalles parciales del contenido buscado.
El problema principal es que los sistemas de búsqueda tradicionales dependen exclusivamente de nombres de archivo exactos o metadatos específicos, lo cual resulta insuficiente en la gran mayoría de casos cuando se quiere buscar un archivo específico.
Para abordar este problema, se ha desarrollado una solución basada en Inteligencia Artificial multimodal, capaz de procesar simultáneamente textos, imágenes y otro tipo de formatos más avanzados como audio o vídeo.
La arquitectura implementada sigue el paradigma Retrieval-Augmented Generation (RAG) el cual organiza el proceso de búsqueda en 2 partes. Primero, se recuperan los archivos más relevantes mediante embeddings vectoriales a partir de la consulta del usuario sobre la base de datos. Posteriormente, se genera una respuesta adaptada utilizando modelos de lenguaje natural y, en este caso, un prompt específico que permite filtrar y modificar, en caso de ser necesario, dicha respuesta de la base de datos.
El proyecto se ha estructurado siguiendo una adaptación simplificada de la metodología Scrum, dividiendo el trabajo en sprints iterativos de aproximadamente 2 semanas.
La implementación técnica combina diversas herramientas modernas. Vue,js proporciona la interfaz de usuario mientras que Flask en Python gestiona la API REST del backend. Prefect se encarga de la orquestación de tareas y LMStudio facilita la ejecución local de modelos de lenguaje cuantizados. Esta arquitectura modular garantiza escalabilidad y facilidad de mantenimiento.
Una característica importante del sistema es su capacidad de poder ejecutarse de manera completamente local, de esta manera el usuario puede utilizar este sistema de forma privada y segura. Además, si el usuario no dispusiera de un dispositivo con suficiente rendimiento para utilizar los modelos en local estos podrían usarse desde la nube. Pero la idea de que el sistema sea modular es que el usuario pueda utilizar un modelo pequeño y optimizado para su dispositivo.
Las evaluaciones realizadas demuestran que el sistema logra identificar archivos relevantes mediante consultas en lenguaje natural, ofreciendo resultados considerablemente más precisos que los métodos de búsqueda tradicionales. El rendimiento se mantiene estable incluso en dispositivos de uso doméstico, pero esto depende de los modelos utilizados.
Los resultados obtenidos confirman que aplicar Inteligencia Artificial sobre este problema es viable. El proyecto no solo ofrece una solución funcional a un problema cotidiano, sino que también establece fundamentos para futuras investigaciones en búsqueda multimedia inteligente.
En conclusión, este proyecto demuestra como la Inteligencia Artificial puede ayudar de manera considerable en la búsqueda de contenido multimedia, acercando más la tecnología al ámbito doméstico y cotidiano de los usuarios.

\cleardoublepage %salta a nueva página impar
\chapter*{Agradecimientos}
\thispagestyle{empty}
Este rinconcito es para vosotros, para toda esa gente increíble que ha hecho posible que hoy esté aquí, escribiendo estas líneas.

Primero, a ese montón de compañeros y amigos que me he cruzado en la carrera. He aprendido un millón de cosas con vosotros, no solo de manera académica, sino lecciones de vida que se quedan para siempre. Sin vuestras risas, vuestros ánimos en los momentos de bajón y esa forma de tirar para adelante juntos, no sé si habría encontrado las fuerzas para seguir tantas veces. Sois, en gran parte, la razón de que esté celebrando este logro.

A mi familia, mi pilar fundamental. Gracias por creer en mí incluso cuando yo dudaba, por ponerme las cosas fáciles y por todo el apoyo para que pudiera dedicarme a esto. Sois increíbles. Y un gracias enorme y especial para mi hermano, Abel Gandía Ruiz. Tú fuiste quien me abrió los ojos a este mundo tan interesante de la programación cuando yo no tenía ni idea, quien me animó y me echó una mano para empezar.

También quiero acordarme de mis profes. Algunos habéis sido una inspiración, de esos que te contagian la ilusión y te hacen descubrir la magia en sitios donde nunca te lo hubieras imaginado. Gracias por inspirarme y ayudarme a ser un mejor ingeniero.

Y, cómo no, a mi tutor, Iván Gadea Sáez. Gracias por guiarme con este proyecto, por tu paciencia infinita y por ayudarme a calmar todos los nervios y dudas que me han ido surgiendo.

De verdad, a todos y cada uno, ¡muchísimas gracias!

\cleardoublepage %salta a nueva página impar
% Aquí va la dedicatoria si la hubiese. Si no, comentar la(s) linea(s) siguientes
\chapter*{}
\setlength{\leftmargin}{0.5\textwidth}
\setlength{\parsep}{0cm}
\addtolength{\topsep}{0.5cm}
\begin{flushright}
\small\em{
A quienes me inspiraron a soñar y a programar, recordándome que,\\
si puedo imaginarlo, puedo crearlo. \footnote{Alejandro Taboada, creador del canal "Programación ATS"}
}
\end{flushright}
