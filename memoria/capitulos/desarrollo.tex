%%%%%%%%%%%%%%%%%%%%%%%%%%%%%%%%%%%%%%%%%%%%%%%%%%%%%%%%%%%%%%%%%%%%%%%%
% Plantilla TFG/TFM
% Escuela Politécnica Superior de la Universidad de Alicante
% Realizado por: Jose Manuel Requena Plens
% Contacto: info@jmrplens.com / Telegram:@jmrplens
%%%%%%%%%%%%%%%%%%%%%%%%%%%%%%%%%%%%%%%%%%%%%%%%%%%%%%%%%%%%%%%%%%%%%%%%

\chapter{Análisis, Especificación y diseño}
\label{analisis}

\section{Requisitos del sistema}
En esta sección se detallan los requisitos del sistema, divididos en requisitos funcionales, no funcionales y de configuración. Los requisitos se han estructurado en formato tabular para facilitar su comprensión y seguimiento durante el desarrollo del proyecto.

\subsection{Requisitos funcionales}
Los requisitos funcionales describen el comportamiento que debe tener el sistema, las funcionalidades que debe ofrecer y las operaciones que debe realizar.

\begin{table}[H]
\centering
\begin{tabular}{|p{1cm}|p{4cm}|p{9cm}|}
\hline
\textbf{ID} & \textbf{Nombre} & \textbf{Descripción} \\
\hline
RF-01 & Detección de ficheros & El sistema debe detectar nuevos ficheros en el directorio observado. \\
\hline
RF-02 & Diferenciación de tipos & El sistema debe diferenciar el tipo de archivo a analizar (texto, imagen, vídeo, audio, otros). \\
\hline
RF-03 & Ejecución de modelos & El sistema debe ejecutar el modelo correspondiente que extraerá la información del fichero a la base de datos. \\
\hline
RF-04 & Almacenamiento & El sistema debe almacenar todos los datos posibles sobre el fichero analizado en una base de datos. \\
\hline
RF-05 & Interfaz web & El sistema debe tener una interfaz web super-simple donde el usuario podrá escribir su consulta en lenguaje natural y darle a un botón para realizar la búsqueda. \\
\hline
RF-06 & Resultados de búsqueda & El sistema responderá con un conjunto de resultados potencialmente interesantes a partir de la consulta de búsqueda, ordenados de más a menos "interesante". \\
\hline
RF-07 & Entrada por línea de comandos & El sistema debe tener una entrada por línea de comandos (ej: \texttt{LLMSearch --query "mapa del mundo en el que hay marcados los mejores parques naturales"}). \\
\hline
RF-08 & Presentación de resultados & El resultado será la ruta del fichero junto a una pequeña descripción del mismo (enlaces clicables al fichero y a la carpeta que lo contiene). \\
\hline
RF-09 & Inspección de archivos comprimidos & Los ficheros comprimidos deberían poder inspeccionarse por dentro. \\
\hline
RF-10 & Tipos de ficheros a procesar & El sistema debe procesar los siguientes tipos de ficheros: \\
& & - Documentos de texto \\
& & - Imágenes \\
& & - Vídeos \\
& & - Ficheros de sonido \\
& & - Otros (bases de datos, ejecutables, etc.) \\
\hline
\end{tabular}
\caption{Requisitos funcionales del sistema}
\label{tab:req_funcionales}
\end{table}

\subsection{Requisitos no funcionales}
Los requisitos no funcionales especifican criterios que pueden usarse para juzgar la operación de un sistema en lugar de sus comportamientos específicos.

\begin{table}[H]
\centering
\begin{tabular}{|p{1cm}|p{4cm}|p{9cm}|}
\hline
\textbf{ID} & \textbf{Nombre} & \textbf{Descripción} \\
\hline
RNF-01 & Configuración web & La web debe tener una pequeña parte de configuración discreta pero accesible en todo momento. \\
\hline
RNF-02 & Arquitectura modular & La arquitectura se debe dividir en un "buscador" y un "explorador" y deben ser completamente separadas para poder ser reutilizadas. \\
\hline
RNF-03 & Ejecución sin GPU & El sistema debe poder ejecutarse en un ordenador sin GPU (opcional). \\
\hline
RNF-04 & Parámetro de consulta & La entrada por línea de comandos aceptará un parámetro \texttt{--query} junto al término de búsqueda. \\
\hline
RNF-05 & Resultados en CLI & La entrada por línea de comandos devolverá los resultados de la misma manera que el buscador web con la diferencia de que solo devolverá información adicional si se le añade el parámetro \texttt{--verbose}. \\
\hline
RNF-06 & Estado del sistema & La entrada por línea de comandos tendrá un parámetro \texttt{--status} que devolverá el estado del sistema: número de archivos procesados sobre el número total de archivos en observación, cantidad de ficheros de cada tipo, errores encontrados... \\
\hline
RNF-07 & Configuración por CLI & Se añadirán los parámetros necesarios para poder configurar el sistema desde línea de comandos. \\
\hline
\end{tabular}
\caption{Requisitos no funcionales del sistema}
\label{tab:req_no_funcionales}
\end{table}

\subsection{Requisitos de configuración}
Los requisitos de configuración especifican las opciones que el usuario debe poder ajustar en el sistema.

\begin{table}[H]
\centering
\begin{tabular}{|p{1cm}|p{4cm}|p{9cm}|}
\hline
\textbf{ID} & \textbf{Nombre} & \textbf{Descripción} \\
\hline
RC-01 & Directorio de observación & Directorio donde se están observando nuevos ficheros. \\
\hline
RC-02 & Regulación de carga & Regular la carga (limitar la CPU al X\%). \\
\hline
RC-03 & Tipo de modelo LLM & Tipo de modelo LLM a utilizar (Local \textit{(LLM Studio)} ó en la nube). \\
\hline
RC-04 & Búsqueda por imagen & Posibilidad de poner una foto de una persona y que la busque en los ficheros. \\
\hline
\end{tabular}
\caption{Requisitos de configuración del sistema}
\label{tab:req_configuracion}
\end{table}

\section{Diagrama de arquitectura del sistema}
A continuación se presenta un diagrama de la arquitectura del sistema que muestra la división entre el "buscador" y el "explorador" según el requisito no funcional RNF-02.

% Aquí se podría incluir un diagrama creado con TikZ o importado como imagen

\section{Casos de uso}
Los casos de uso describen las interacciones típicas entre los usuarios y el sistema, mostrando cómo se utilizarían las funcionalidades principales.

% Aquí se podrían describir algunos casos de uso principales, como:
% - Búsqueda web de contenido
% - Uso de la línea de comandos
% - Configuración del sistema
% - etc.