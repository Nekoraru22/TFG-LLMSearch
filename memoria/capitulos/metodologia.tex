%%%%%%%%%%%%%%%%%%%%%%%%%%%%%%%%%%%%%%%%%%%%%%%%%%%%%%%%%%%%%%%%%%%%%%%%
% Plantilla TFG/TFM
% Escuela Politécnica Superior de la Universidad de Alicante
% Realizado por: Jose Manuel Requena Plens
% Contacto: info@jmrplens.com / Telegram:@jmrplens
%%%%%%%%%%%%%%%%%%%%%%%%%%%%%%%%%%%%%%%%%%%%%%%%%%%%%%%%%%%%%%%%%%%%%%%%

\chapter{Metodología}
\label{metodologia}

\section{Organización}

La metodología que se va a seguir es Scrum adaptado simplificado.
Scrum es una metodología ágil que facilita la gestión de proyectos complejos a través de equipos autoorganizados que trabajan en ciclos de desarrollo (sprints) de forma iterativa e incrementa y que promueve la entrega continua de valor.
Dado que el proyecto se realizará por un único estudiante, este asumirá el rol de Product Owner, Desarrollador y Scrum Master, mientras que el tutor actuará como cliente y pondrá los requisitos y los posibles cambios a lo largo del proyecto.
Se harán sprints de aproximadamente 2 semanas desde el inicio del cuatrimestre donde se verán avances, se resolverán dudas y se definirán los objetivos del siguiente sprint.

\section{Apartado técnico}
\label{sec:apartado_tecnico}

Para la ejecución y desarrollo del presente \gls{tfg}, se ha dispuesto del siguiente entorno técnico, tanto a nivel de hardware como de software. Esta configuración ha sido la base sobre la cual se han realizado todas las pruebas, desarrollos y validaciones del sistema propuesto.

\subsection{Equipamiento Hardware}
El equipo informático utilizado para el desarrollo del proyecto cuenta con las siguientes especificaciones:
\begin{itemize}
    \item \textbf{Procesador (CPU):} AMD Ryzen 9 7900X3D 4.4GHz/5.6GHz
    \item \textbf{Memoria RAM:} Corsair Vengeance RGB DDR5 6000MHz 64GB 2x32GB CL30
    \item \textbf{Tarjeta Gráfica (GPU):} RTX 4070 Ti SUPER Trinity 16GB GDDR6X
    \item \textbf{Almacenamiento:} SSD NVMe Samsung 970 EVO Plus de 1TB
    \item \textbf{Sistema Operativo Principal:} Windows 11 Pro / Ubuntu 22.04 LTS
\end{itemize}

\subsection{Software y Herramientas de Desarrollo}
El desarrollo del software se ha apoyado en las siguientes herramientas, lenguajes y bibliotecas:
\begin{itemize}
    \item \textbf{Entorno de Desarrollo Integrado (IDE):} Visual Studio Code
    \item \textbf{Control de Versiones:} Git con repositorios alojados en GitHub
    \item \textbf{Gestor de Entornos y Paquetes:} pip de Python
    \item \textbf{Navegador Web (para pruebas de interfaz):} Google Chrome y OperaGX en sus versiones más recientes
    \item \textbf{Herramientas de Documentación:} LaTeX (MiKTeX), Excalidraw (para diagramas)
\end{itemize}