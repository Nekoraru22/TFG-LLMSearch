%%%%%%%%%%%%%%%%%%%%%%%%%%%%%%%%%%%%%%%%%%%%%%%%%%%%%%%%%%%%%%%%%%%%%%%%
% Plantilla TFG/TFM
% Escuela Politécnica Superior de la Universidad de Alicante
% Realizado por: Jose Manuel Requena Plens
% Contacto: info@jmrplens.com / Telegram:@jmrplens
%%%%%%%%%%%%%%%%%%%%%%%%%%%%%%%%%%%%%%%%%%%%%%%%%%%%%%%%%%%%%%%%%%%%%%%%

\chapter{Metodología}
\label{metodologia}

\section{Organización}

La metodología que se va a seguir es Scrum Adaptado (con un toque de Kanban).
Scrum es una metodología ágil que facilita la gestión de proyectos complejos a través de equipos autoorganizados que trabajan en ciclos de desarrollo (sprints) de forma iterativa e incrementa y que promueve la entrega continua de valor.
La adaptación en este caso consiste en que yo asumiré el rol de Product Owner, Desarrollador y Scrum Master, mientras mi tutor actúa como cliente (requisitos).
Se harán sprints de aproximadamente 2 semanas desde el inicio del cuatrimestre donde se verán avances, se resolverán dudas y se definirán los objetivos del siguiente sprint.

\section{Apartado técnico}

Explicar que cosas he elegido y por qué + my hardware